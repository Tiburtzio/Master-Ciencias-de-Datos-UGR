\documentclass[]{article}
\usepackage{lmodern}
\usepackage{amssymb,amsmath}
\usepackage{ifxetex,ifluatex}
\usepackage{fixltx2e} % provides \textsubscript
\ifnum 0\ifxetex 1\fi\ifluatex 1\fi=0 % if pdftex
  \usepackage[T1]{fontenc}
  \usepackage[utf8]{inputenc}
\else % if luatex or xelatex
  \ifxetex
    \usepackage{mathspec}
  \else
    \usepackage{fontspec}
  \fi
  \defaultfontfeatures{Ligatures=TeX,Scale=MatchLowercase}
\fi
% use upquote if available, for straight quotes in verbatim environments
\IfFileExists{upquote.sty}{\usepackage{upquote}}{}
% use microtype if available
\IfFileExists{microtype.sty}{%
\usepackage{microtype}
\UseMicrotypeSet[protrusion]{basicmath} % disable protrusion for tt fonts
}{}
\usepackage[margin=1in]{geometry}
\usepackage{hyperref}
\hypersetup{unicode=true,
            pdftitle={Ejercicios R Input/Output},
            pdfauthor={Alberto Armijo Ruiz},
            pdfborder={0 0 0},
            breaklinks=true}
\urlstyle{same}  % don't use monospace font for urls
\usepackage{color}
\usepackage{fancyvrb}
\newcommand{\VerbBar}{|}
\newcommand{\VERB}{\Verb[commandchars=\\\{\}]}
\DefineVerbatimEnvironment{Highlighting}{Verbatim}{commandchars=\\\{\}}
% Add ',fontsize=\small' for more characters per line
\usepackage{framed}
\definecolor{shadecolor}{RGB}{248,248,248}
\newenvironment{Shaded}{\begin{snugshade}}{\end{snugshade}}
\newcommand{\KeywordTok}[1]{\textcolor[rgb]{0.13,0.29,0.53}{\textbf{#1}}}
\newcommand{\DataTypeTok}[1]{\textcolor[rgb]{0.13,0.29,0.53}{#1}}
\newcommand{\DecValTok}[1]{\textcolor[rgb]{0.00,0.00,0.81}{#1}}
\newcommand{\BaseNTok}[1]{\textcolor[rgb]{0.00,0.00,0.81}{#1}}
\newcommand{\FloatTok}[1]{\textcolor[rgb]{0.00,0.00,0.81}{#1}}
\newcommand{\ConstantTok}[1]{\textcolor[rgb]{0.00,0.00,0.00}{#1}}
\newcommand{\CharTok}[1]{\textcolor[rgb]{0.31,0.60,0.02}{#1}}
\newcommand{\SpecialCharTok}[1]{\textcolor[rgb]{0.00,0.00,0.00}{#1}}
\newcommand{\StringTok}[1]{\textcolor[rgb]{0.31,0.60,0.02}{#1}}
\newcommand{\VerbatimStringTok}[1]{\textcolor[rgb]{0.31,0.60,0.02}{#1}}
\newcommand{\SpecialStringTok}[1]{\textcolor[rgb]{0.31,0.60,0.02}{#1}}
\newcommand{\ImportTok}[1]{#1}
\newcommand{\CommentTok}[1]{\textcolor[rgb]{0.56,0.35,0.01}{\textit{#1}}}
\newcommand{\DocumentationTok}[1]{\textcolor[rgb]{0.56,0.35,0.01}{\textbf{\textit{#1}}}}
\newcommand{\AnnotationTok}[1]{\textcolor[rgb]{0.56,0.35,0.01}{\textbf{\textit{#1}}}}
\newcommand{\CommentVarTok}[1]{\textcolor[rgb]{0.56,0.35,0.01}{\textbf{\textit{#1}}}}
\newcommand{\OtherTok}[1]{\textcolor[rgb]{0.56,0.35,0.01}{#1}}
\newcommand{\FunctionTok}[1]{\textcolor[rgb]{0.00,0.00,0.00}{#1}}
\newcommand{\VariableTok}[1]{\textcolor[rgb]{0.00,0.00,0.00}{#1}}
\newcommand{\ControlFlowTok}[1]{\textcolor[rgb]{0.13,0.29,0.53}{\textbf{#1}}}
\newcommand{\OperatorTok}[1]{\textcolor[rgb]{0.81,0.36,0.00}{\textbf{#1}}}
\newcommand{\BuiltInTok}[1]{#1}
\newcommand{\ExtensionTok}[1]{#1}
\newcommand{\PreprocessorTok}[1]{\textcolor[rgb]{0.56,0.35,0.01}{\textit{#1}}}
\newcommand{\AttributeTok}[1]{\textcolor[rgb]{0.77,0.63,0.00}{#1}}
\newcommand{\RegionMarkerTok}[1]{#1}
\newcommand{\InformationTok}[1]{\textcolor[rgb]{0.56,0.35,0.01}{\textbf{\textit{#1}}}}
\newcommand{\WarningTok}[1]{\textcolor[rgb]{0.56,0.35,0.01}{\textbf{\textit{#1}}}}
\newcommand{\AlertTok}[1]{\textcolor[rgb]{0.94,0.16,0.16}{#1}}
\newcommand{\ErrorTok}[1]{\textcolor[rgb]{0.64,0.00,0.00}{\textbf{#1}}}
\newcommand{\NormalTok}[1]{#1}
\usepackage{graphicx,grffile}
\makeatletter
\def\maxwidth{\ifdim\Gin@nat@width>\linewidth\linewidth\else\Gin@nat@width\fi}
\def\maxheight{\ifdim\Gin@nat@height>\textheight\textheight\else\Gin@nat@height\fi}
\makeatother
% Scale images if necessary, so that they will not overflow the page
% margins by default, and it is still possible to overwrite the defaults
% using explicit options in \includegraphics[width, height, ...]{}
\setkeys{Gin}{width=\maxwidth,height=\maxheight,keepaspectratio}
\IfFileExists{parskip.sty}{%
\usepackage{parskip}
}{% else
\setlength{\parindent}{0pt}
\setlength{\parskip}{6pt plus 2pt minus 1pt}
}
\setlength{\emergencystretch}{3em}  % prevent overfull lines
\providecommand{\tightlist}{%
  \setlength{\itemsep}{0pt}\setlength{\parskip}{0pt}}
\setcounter{secnumdepth}{0}
% Redefines (sub)paragraphs to behave more like sections
\ifx\paragraph\undefined\else
\let\oldparagraph\paragraph
\renewcommand{\paragraph}[1]{\oldparagraph{#1}\mbox{}}
\fi
\ifx\subparagraph\undefined\else
\let\oldsubparagraph\subparagraph
\renewcommand{\subparagraph}[1]{\oldsubparagraph{#1}\mbox{}}
\fi

%%% Use protect on footnotes to avoid problems with footnotes in titles
\let\rmarkdownfootnote\footnote%
\def\footnote{\protect\rmarkdownfootnote}

%%% Change title format to be more compact
\usepackage{titling}

% Create subtitle command for use in maketitle
\newcommand{\subtitle}[1]{
  \posttitle{
    \begin{center}\large#1\end{center}
    }
}

\setlength{\droptitle}{-2em}

  \title{Ejercicios R Input/Output}
    \pretitle{\vspace{\droptitle}\centering\huge}
  \posttitle{\par}
    \author{Alberto Armijo Ruiz}
    \preauthor{\centering\large\emph}
  \postauthor{\par}
      \predate{\centering\large\emph}
  \postdate{\par}
    \date{23 de octubre de 2018}


\begin{document}
\maketitle

\subsection{Ejercicios Input/Output}\label{ejercicios-inputoutput}

\subsubsection{1. Pide al usuario que introduzca un string s y un número
n y que muestre en pantalla n veces seguidas el string s (sin espacios
entre palabra y
palabra).}\label{pide-al-usuario-que-introduzca-un-string-s-y-un-numero-n-y-que-muestre-en-pantalla-n-veces-seguidas-el-string-s-sin-espacios-entre-palabra-y-palabra.}

\begin{Shaded}
\begin{Highlighting}[]
\KeywordTok{print}\NormalTok{(}\StringTok{"Introduzca un string:"}\NormalTok{)}
\end{Highlighting}
\end{Shaded}

\begin{verbatim}
## [1] "Introduzca un string:"
\end{verbatim}

\begin{Shaded}
\begin{Highlighting}[]
\NormalTok{m_string =}\StringTok{ }\KeywordTok{scan}\NormalTok{(}\StringTok{""}\NormalTok{,}\DataTypeTok{what=}\KeywordTok{character}\NormalTok{())}
\KeywordTok{print}\NormalTok{(}\StringTok{"Introduzca un número:"}\NormalTok{)}
\end{Highlighting}
\end{Shaded}

\begin{verbatim}
## [1] "Introduzca un número:"
\end{verbatim}

\begin{Shaded}
\begin{Highlighting}[]
\NormalTok{m_number =}\StringTok{ }\KeywordTok{scan}\NormalTok{(}\StringTok{""}\NormalTok{,}\DataTypeTok{what=}\KeywordTok{integer}\NormalTok{())}
\NormalTok{m_list =}\StringTok{ }\KeywordTok{rep}\NormalTok{(m_string,m_number)}
\KeywordTok{paste}\NormalTok{(m_list,}\DataTypeTok{collapse=} \StringTok{''}\NormalTok{)}
\end{Highlighting}
\end{Shaded}

\begin{verbatim}
## [1] ""
\end{verbatim}

\subsubsection{2. Crea tres ficheros llamados dos.txt, tres.txt y
cinco.txt que contenga la tabla del 2, la del 3 y la del 5
respectivamente (los primeros 10 valores de cada tabla, un número en
cada línea de cada
fichero).}\label{crea-tres-ficheros-llamados-dos.txt-tres.txt-y-cinco.txt-que-contenga-la-tabla-del-2-la-del-3-y-la-del-5-respectivamente-los-primeros-10-valores-de-cada-tabla-un-numero-en-cada-linea-de-cada-fichero.}

\begin{Shaded}
\begin{Highlighting}[]
\NormalTok{unoaldiez=}\DecValTok{1}\OperatorTok{:}\DecValTok{10}
\KeywordTok{write.table}\NormalTok{(}\KeywordTok{matrix}\NormalTok{(unoaldiez}\OperatorTok{*}\DecValTok{2}\NormalTok{,}\DataTypeTok{nrow=}\DecValTok{10}\NormalTok{),}\StringTok{"dos.txt"}\NormalTok{,}\DataTypeTok{row.names =}\NormalTok{ F,}\DataTypeTok{col.names =}\NormalTok{ F)}
\KeywordTok{write.table}\NormalTok{(}\KeywordTok{matrix}\NormalTok{(unoaldiez}\OperatorTok{*}\DecValTok{3}\NormalTok{,}\DataTypeTok{nrow=}\DecValTok{10}\NormalTok{),}\StringTok{"tres.txt"}\NormalTok{,}\DataTypeTok{row.names =}\NormalTok{ F, }\DataTypeTok{col.names =}\NormalTok{ F)}
\KeywordTok{write.table}\NormalTok{(}\KeywordTok{matrix}\NormalTok{(unoaldiez}\OperatorTok{*}\DecValTok{5}\NormalTok{,}\DataTypeTok{nrow=}\DecValTok{10}\NormalTok{),}\StringTok{"cinco.txt"}\NormalTok{,}\DataTypeTok{row.names =}\NormalTok{ F, }\DataTypeTok{col.names =}\NormalTok{ F)}
\end{Highlighting}
\end{Shaded}

\subsubsection{3. Carga los tres ficheros creados en el punto anterior y
construye una matriz que, en cada columna, tengo el contenido de cada
fichero.}\label{carga-los-tres-ficheros-creados-en-el-punto-anterior-y-construye-una-matriz-que-en-cada-columna-tengo-el-contenido-de-cada-fichero.}

\begin{Shaded}
\begin{Highlighting}[]
\NormalTok{tablados =}\StringTok{ }\KeywordTok{scan}\NormalTok{(}\StringTok{"dos.txt"}\NormalTok{)}
\NormalTok{tablatres =}\StringTok{ }\KeywordTok{scan}\NormalTok{(}\StringTok{"tres.txt"}\NormalTok{)}
\NormalTok{tablacinco =}\StringTok{ }\KeywordTok{scan}\NormalTok{(}\StringTok{"cinco.txt"}\NormalTok{)}
\NormalTok{m =}\StringTok{ }\KeywordTok{cbind}\NormalTok{(tablados,tablatres,tablacinco); m}
\end{Highlighting}
\end{Shaded}

\begin{verbatim}
##       tablados tablatres tablacinco
##  [1,]        2         3          5
##  [2,]        4         6         10
##  [3,]        6         9         15
##  [4,]        8        12         20
##  [5,]       10        15         25
##  [6,]       12        18         30
##  [7,]       14        21         35
##  [8,]       16        24         40
##  [9,]       18        27         45
## [10,]       20        30         50
\end{verbatim}

\subsubsection{4. Escribe las cinco primera filas de matriz del
ejercicio anterior en un fichero nuevo llamado prime.txt y las cinco
últimas en otro fichero llamado fin.txt. Ambos ficheros deben tener los
datos separados por
comas.}\label{escribe-las-cinco-primera-filas-de-matriz-del-ejercicio-anterior-en-un-fichero-nuevo-llamado-prime.txt-y-las-cinco-ultimas-en-otro-fichero-llamado-fin.txt.-ambos-ficheros-deben-tener-los-datos-separados-por-comas.}

\begin{Shaded}
\begin{Highlighting}[]
\KeywordTok{write.csv}\NormalTok{(m[}\DecValTok{1}\OperatorTok{:}\DecValTok{5}\NormalTok{,],}\StringTok{"prime.txt"}\NormalTok{,}\DataTypeTok{row.names =}\NormalTok{ F)}
\KeywordTok{write.csv}\NormalTok{(m[}\DecValTok{6}\OperatorTok{:}\DecValTok{10}\NormalTok{,],}\StringTok{"fin.txt"}\NormalTok{,}\DataTypeTok{row.names =}\NormalTok{ F)}
\end{Highlighting}
\end{Shaded}

\subsubsection{\texorpdfstring{5. Dados dos números introducidos por el
usuario f y c, crea un cuadrado de f filas y c columnas con el caracter
``x''. Un ejemplo con f=4 y c=3
sería:}{5. Dados dos números introducidos por el usuario f y c, crea un cuadrado de f filas y c columnas con el caracter x. Un ejemplo con f=4 y c=3 sería:}}\label{dados-dos-numeros-introducidos-por-el-usuario-f-y-c-crea-un-cuadrado-de-f-filas-y-c-columnas-con-el-caracter-x.-un-ejemplo-con-f4-y-c3-seria}

xxx \newline
xxx \newline
xxx \newline
xxx \newline

\begin{Shaded}
\begin{Highlighting}[]
\KeywordTok{print}\NormalTok{(}\StringTok{"Introduzca filas:"}\NormalTok{)}
\end{Highlighting}
\end{Shaded}

\begin{verbatim}
## [1] "Introduzca filas:"
\end{verbatim}

\begin{Shaded}
\begin{Highlighting}[]
\NormalTok{f =}\StringTok{ }\KeywordTok{scan}\NormalTok{(}\StringTok{""}\NormalTok{,}\DataTypeTok{what =} \KeywordTok{integer}\NormalTok{())}
\KeywordTok{print}\NormalTok{(}\StringTok{"Introduzca columnas:"}\NormalTok{)}
\end{Highlighting}
\end{Shaded}

\begin{verbatim}
## [1] "Introduzca columnas:"
\end{verbatim}

\begin{Shaded}
\begin{Highlighting}[]
\NormalTok{c =}\StringTok{ }\KeywordTok{scan}\NormalTok{(}\StringTok{""}\NormalTok{,}\DataTypeTok{what=}\KeywordTok{integer}\NormalTok{())}
\KeywordTok{cat}\NormalTok{(}\KeywordTok{rep}\NormalTok{(}\KeywordTok{paste}\NormalTok{(}\KeywordTok{rep}\NormalTok{(}\StringTok{"x"}\NormalTok{,}\DecValTok{3}\NormalTok{),}\DataTypeTok{collapse =} \StringTok{''}\NormalTok{),}\DecValTok{4}\NormalTok{),}\DataTypeTok{sep =} \StringTok{'}\CharTok{\textbackslash{}n}\StringTok{'}\NormalTok{)}
\end{Highlighting}
\end{Shaded}

\begin{verbatim}
## xxx
## xxx
## xxx
## xxx
\end{verbatim}

\begin{Shaded}
\begin{Highlighting}[]
\CommentTok{#cat(rep(paste(rep("x",c),collapse = ''),f),sep = '\textbackslash{}n')}
\end{Highlighting}
\end{Shaded}


\end{document}
