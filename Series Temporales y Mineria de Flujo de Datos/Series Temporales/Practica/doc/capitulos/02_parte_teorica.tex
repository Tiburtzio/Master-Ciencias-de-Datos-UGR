\chapter{Parte Teórica}
En este apartado se describirán los conceptos teóricos utilizados en la práctica.
\section{Preprocesamiento}
El preprocesamiento utilizado en la práctica ha sido la imputación de valores; este tipo de preprocesamiento se utiliza en todo tipo de problemas de minería de datos. La imputación de valores consiste en reemplazar valores pérdidos en los datos por datos coherentes. \newline

Otro tipo de preprocesamiento utilizado en la primera parte de la práctica ha sido el filtrado de series temporales; para ello se ha utilizado un filtro de convolución que realiza la media de los quince días más cercanos a cada día; de esta forma se puede conseguir una serie más limpia sin picos que puedan ser considerados ruidosos. El filtrado en series temporales también se puede utilizar para detectar ciclos o tendencias dentro de series con estacionalidad.
\section{Análisis de tendencia y estacionalidad}
Para el análisis de tendencia y estacionalidad se ha utilizado descomposición de series temporales. Existen varios métodos de descomposición de series temporales; como por ejemplo la descomposición mediante \textit{Medias Móviles} (\textit{Moving Averages} en inglés, \textit{MA}) o descomposición \textit{STL}. \newline

La descomposición de series temporales consiste en la separación de los datos de una serie temporal en tres componentes: tendencia, estacionalidad y residuos. La tendencia expresa incrementos o decrementos durante largos periodos de tiempos, estos incrementos/decrementos no tienen porque se de tipo lineal. La componente estacional muestra cambios en los datos afectados por un patrón estacional como por ejemplo las estaciones del año. La componente residual expresa el resto de la serie temporal al eliminar los valores de las dos componentes anteriores. Existen dos tipos de descomposiciones: aditiva y multiplicativa. \newline

Para realizar una descomposición mediante \textit{Medias Móviles} se debe seguir el siguiente proceso: utilizar el método de \textit{Medias Móviles} para calcular la componentes de tendencia y calcular la serie sin tendencia restando/dividiendo el resultado del algoritmo de \textit{Medias Móviles}. Tras esto, se calcula la componente estacional como la media de los valores de un ciclo dado y después ajustada a 0. Por último, se calcula la componente residual como la resta/división de la suma/multiplicación de las otras dos componentes.\newline

Aparte de la descomposición de series temporales, existen otras formas de estudiar la tendencia de una serie; como por ejemplo el filtrado (mediante \textit{Medias Móviles} por ejemplo) o la estimación funcional, es decir, estimar la tendencia mediante una función, bien lineal o no lineal.
\section{Estacionariedad}
La estacionariedad es una característica de algunas series temporales para las cuales sus propiedades no dependen del momento en el que se observa, esto significa que la varianza en los datos es constante. Para detectar la estacionariedad en las series se suele utilizar el gráfico \textit{ACF}; ya que las series estacionarias suelen tener gráficos \textit{ACF} que descienden rápidamente a 0.\newline

La gráfica \textit{ACF} es un gráfico que representa la autocorrelación entre los distintos estados de la serie, es decir, muestra la importación de los estados anteriores de la serie con el estado actual y nos indica hasta que estado tiene importancia los valores de la serie para predecir estados siguientes.\newline

Con la gráfica \textit{ACF} no es suficiente para determinar si una serie es estacionaria, ya que series con estacionalidad pueden mostrar también gráficas \textit{ACF} que descienden rápidamente a 0. Por ello, es necesario también utilizar un test que nos indique si la serie es estacionaria, o necesita ser diferenciada. La diferenciación entre series temporales se trata del cálculo entre observaciones sucesivas. El test que se debe utilizar para esto es el test de \textit{Dickey-Fuller Ampliado}. Si los datos de la serie pasan este test,  se sabe que la serie es estacionaria; en caso contrario, se sabe que se debe diferenciar la serie para conseguir estacionariedad. \newline

La estacionariedad en series no estacionarias se puede obtener mediante la eliminación de sus componentes de tendencia y estacionalidad (si es que tienen) y la diferenciación de la serie.
\section{Modelado de series temporales}
Para el modelado de series temporales existen diferentes metodologías; una de la más comunes es el uso de modelos \textit{ARIMA} con series estacionarias.\newline

Los modelos \textit{ARIMA} son modelos formados por la combinación de diferenciación en la serie, \textit{modelos autoregresivos} y \textit{modelos de medias móviles} de diferentes grados. Los \textit{modelos autoregresivos} se calculan como una combinación lineal de \textit{p} estados anteriores de la serie y un error; donde \textit{p} representa el grado del modelo. Los \textit{modelos de medias móviles} se calculan como la combinación lineal de \textit{q} coeficientes en estados anteriores y sus errores asociados, donde \textit{q} representa el grado del modelo. Por lo tanto, los modelos \textit{ARIMA} pueden ser una combinación de un \textit{modelo autoregresivo} de grado \textit{p}, un \textit{modelo de medias móviles} de grado \textit{q} sobre una serie diferenciada \textit{d} veces. \newline

Para saber cuál es el grado de cada una de estas componentes es necesario utilizar los gráficos \textit{ACF} y \textit{PACF} para determinar el grado de los modelos y el test de \textit{Dickey-Fuller Aumentado} para saber cuantas veces es necesario diferenciar la serie. El gráfico \textit{PACF}, al igual que el gráfico \textit{ACF}, muestra la correlación de los diferentes estados de una serie, pero esta vez tiene en cuenta la diferencia de espacio entre los diferentes estados.\newline

Los modelos \textit{AR} (autoregression), suelen tener gráficas \textit{ACF} que descienden rápida a 0 y gráficas \textit{PACF} que tardan más descender a 0. Los modelos \textit{MA} (Moving Averages), tienen gráficas \textit{PACF} que descienden rápida a 0 y gráficas \textit{ACF} que tardan más en descender. Para algunos casos, es posible que no haya diferencias tan significativas entre ambas gráficas; en esos casos se debe probar también la combinación de ambos (modelo \textit{ARMA}).\newline

Una vez se ha obtenido un modelo, o modelos, se necesitan medidas para elegir el mejor modelo posible. Existen diferentes medidas, como por ejemplo el error cuadrático medio (\textit{MSE}) o el error absoluto medio (\textit{MAE}); el problema de estas medidas que no consideran la complejidad de los modelos, solamente consideran el error cometido por estos. Por ello, también es necesario utilizar medidas como el criterio de Akaike (\textit{AIC}) que también tienen en cuenta la complejidad del modelo y por lo tanto nos puede mostrar cual de los modelos entrenados puede ser más interesante.